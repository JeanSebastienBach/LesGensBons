\documentclass[11pt,letterpaper]{article}
\usepackage{fullpage}
\usepackage[top=1.5cm, bottom=3.5cm, left=1.5cm, right=1.5cm]{geometry}
\usepackage{amsmath,amsthm,amsfonts,amssymb,amscd}
\usepackage{french}
\usepackage[utf8]{inputenc}
\usepackage[T1]{fontenc}
\usepackage{lastpage}
\usepackage{enumerate}
\usepackage{fancyhdr}
\usepackage{mathrsfs}
\usepackage{xcolor}

\usepackage{graphics} %inclusion de figures
\usepackage{graphicx} %inclusion de figures

\usepackage{sidecap}
\sidecaptionvpos{figure}{c}
\usepackage{listings}
\usepackage{hyperref}
\usepackage{numprint}
\usepackage[bottom]{footmisc}
\usepackage[justification=centering]{caption}
\usepackage[french,frenchkw,boxed,ruled,lined]{algorithm2e}
\SetKw{KwDe}{de}
\SetKw{KwOu}{ou}
\SetKw{KwEt}{et}
\SetKw{Kwrenv}{renvoyer}
\SetKwInput{Variables}{Variables}
\SetKwInput{Variable}{Variable}
\usepackage{xifthen}
\usepackage{colortbl}
\definecolor{green}{rgb}{0.0, 0.5, 0.0}
\usepackage{array,multirow}
\usepackage{wrapfig}
\usepackage{bussproofs}
\usepackage{stmaryrd}
\usepackage{tikz}
\usetikzlibrary{positioning}
\definecolor{processblue}{cmyk}{0.96,0,0,0}

\newcommand{\exo}[1]{\Large \textbf{Exercice #1} \vspace{10px} \normalsize}
\newcommand\tab[1][12pt]{\hspace*{#1}}

\newcommand{\titlebox}[2]{%
\tikzstyle{titlebox}=[rectangle,inner sep=10pt,inner ysep=10pt,draw]%
\tikzstyle{title}=[fill=white]%
%
\bigskip\noindent\begin{tikzpicture}
\node[titlebox] (box){%
    \begin{minipage}{0.94\textwidth}
#2
    \end{minipage}
};
%\draw (box.north west)--(box.north east);
\node[title] at (box.north) {#1};
\end{tikzpicture}\bigskip%
}

% INFOS
\author{Antoine AFFLATET, Aurélien BESNIER, Hayaat HEBIRET, Josua PHILIPPOT et Jérémie ROUX\\(L3 Groupe C)}
\title{HLIN602 - Logique - TD Preuve Séquents}
\date{2019 - 2020}

\setlength{\parindent}{0cm}

\begin{document}

\maketitle

\vspace{10 px}

\exo{A(b)}

Montrons que les séquents 1, 2 et 3 avant l'application de leur règle respectivement $\land_{g}$, $\lor_{d}$ et $\neg_{d}$ sont sémantiquement équivalents aux séquents après l'application.

Soit I une interprétation de domaine D et une assignation $\rho$ des variables des séquents dans le domaine D.\\
\begin{prooftree}
{\bfseries (1)}
            \AxiomC{$\Gamma, A , B \vdash \Phi$}
            \RightLabel{\scriptsize$\land_{g}$}
            \UnaryInfC{$A \land B ,\Gamma \vdash \Phi$}
\end{prooftree}

\begin{prooftree}
{\bfseries (2)}
            \AxiomC{$\Gamma\vdash A , B , \Phi$}
            \RightLabel{\scriptsize$\lor_{d}$}
            \UnaryInfC{$\Gamma \vdash A \lor B , \Phi$}
\end{prooftree}

\begin{prooftree}
{\bfseries (3)}
            \AxiomC{$A,\Gamma\vdash\Phi$}
            \RightLabel{\scriptsize$\neg_{d}$}
            \UnaryInfC{$ \Gamma \vdash \neg A ,\Phi$}
\end{prooftree}

\textbf{(1)}\\
$\llbracket A , B ,\Gamma\vdash \Phi \rrbracket^{I}_ {\rho}$\\
$= \llbracket (A \land B \land \Gamma) \Rightarrow \Phi\rrbracket^{I}_ {\rho}$\\
$= \llbracket A \land B \land \Gamma \rrbracket^{I}_ {\rho} \Rightarrow_{B} \llbracket \Phi\rrbracket^{I}_ {\rho}$\\
$= (\llbracket A \rrbracket^{I}_ {\rho} \land_{B} \llbracket B \rrbracket^{I}_ {\rho} \land_{B} \llbracket \Gamma \rrbracket^{I}_ {\rho}) \Rightarrow_{B} \llbracket \Phi \rrbracket^{I}_ {\rho}$\\

\vspace{10px}

$\llbracket A \land B , \Gamma \vdash \Phi \rrbracket^{I}_ {\rho}$\\
$= \llbracket (A \land B \land \Gamma) \Rightarrow \Phi\rrbracket^{I}_ {\rho}$\\
$= \llbracket A \land B \land \Gamma\rrbracket^{I}_ {\rho} \Rightarrow_{B} \llbracket \Phi\rrbracket^{I}_ {\rho}$\\
$= (\llbracket A \rrbracket^{I}_ {\rho} \land_{B} \llbracket B \rrbracket^{I}_ {\rho} \land_{B} \llbracket \Gamma \rrbracket^{I}_ {\rho}) \Rightarrow_{B} \llbracket \Phi \rrbracket^{I}_ {\rho}$\\

\vspace{10 px}

On voit donc que les deux séquents du \textbf{(1)} sont sémantiquement équivalents. La règle $\land_{g}$ ne modifie donc pas la sémantique de la formule. Ainsi, si avant l'application de la règle une formule est valide, elle le sera après l'application.

\vspace{10px}

\textbf{(2)}\\
$\llbracket \Gamma\vdash A,B,\Phi \rrbracket^{I}_ {\rho}$\\
$= \llbracket \Gamma \Rightarrow (A \lor B \lor \Phi)\rrbracket^{I}_ {\rho}$\\
$= \llbracket \Gamma\rrbracket^{I}_ {\rho} \Rightarrow_{B} \llbracket A \lor B \lor \Phi\rrbracket^{I}_ {\rho}$\\
$= \llbracket \Gamma \rrbracket^{I}_ {\rho} \Rightarrow_{B} (\llbracket A \rrbracket^{I}_ {\rho} \lor_{B} \llbracket B \rrbracket^{I}_ {\rho} \lor_{B} \llbracket \Phi\rrbracket^{I}_ {\rho})$\\

\vspace{10 px}

$\llbracket \Gamma\vdash A \lor B,\Phi\rrbracket^{I}_ {\rho}$\\
$= \llbracket \Gamma \Rightarrow (A \lor B \lor \Phi)\rrbracket^{I}_ {\rho}$\\
$= \llbracket \Gamma\rrbracket^{I}_ {\rho} \Rightarrow_{B} \llbracket A \lor B \lor \Phi\rrbracket^{I}_ {\rho}$\\
$= \llbracket \Gamma \rrbracket^{I}_ {\rho} \Rightarrow_{B} (\llbracket A \rrbracket^{I}_ {\rho} \lor_{B} \llbracket B \rrbracket^{I}_ {\rho} \lor_{B} \llbracket \Phi \rrbracket^{I}_ {\rho})$\\

\vspace{10px}

On voit donc que les deux séquents du \textbf{(2)} sont sémantiquement équivalents. La règle $\lor_{d}$ ne modifie donc pas la sémantique de la formule. Ainsi, si avant l'application de la règle une formule est valide, elle le sera après l'application.\\

\vspace{10px}

{\bfseries (3)}\\
$\llbracket A,\Gamma\vdash \Phi \rrbracket^{I}_ {\rho}$\\
$= \llbracket (A \land \Gamma) \Rightarrow \Phi\rrbracket^{I}_ {\rho}$\\
$= \llbracket A \land \Gamma \rrbracket^{I}_ {\rho} \Rightarrow_{B} \llbracket \Phi\rrbracket^{I}_ {\rho}$\\
$= (\llbracket A \rrbracket^{I}_ {\rho} \land_{B} \llbracket \Gamma \rrbracket^{I}_ {\rho} )\Rightarrow_{B} \llbracket \Phi \rrbracket^{I}_ {\rho}$\\
$= \neg_{B} \llbracket A \rrbracket^{I}_ {\rho} \lor_{B} \neg_{B} \llbracket \Gamma \rrbracket^{I}_ {\rho} \lor_{d} \llbracket\Phi \rrbracket^{I}_ {\rho}$\\
\vspace{10px}

$\llbracket \Gamma\vdash \neg A \lor \Phi \rrbracket^{I}_ {\rho}$\\
$= \llbracket \Gamma \Rightarrow \neg  A \lor \Phi\rrbracket^{I}_ {\rho}$\\
$= \llbracket \Gamma \rrbracket^{I}_ {\rho} \Rightarrow_{B} \llbracket \neg  A \lor \Phi\rrbracket^{I}_ {\rho}$\\
$= \llbracket \Gamma \rrbracket^{I}_ {\rho} \Rightarrow_{B} \llbracket \neg A \rrbracket^{I}_ {\rho} \lor_{d} \llbracket\Phi \rrbracket^{I}_ {\rho}$\\
$= \llbracket \Gamma \rrbracket^{I}_ {\rho} \Rightarrow_{B} \neg_{B} \llbracket A \rrbracket^{I}_ {\rho} \lor_{d} \llbracket\Phi \rrbracket^{I}_ {\rho}$\\
$= \neg_{B} \llbracket \Gamma \rrbracket^{I}_ {\rho} \lor_{d} \neg_{B} \llbracket A \rrbracket^{I}_ {\rho} \lor_{B} \llbracket\Phi \rrbracket^{I}_ {\rho}$\\
\vspace{10px}

On voit donc que les deux séquents du \textbf{(3)} sont sémantiquement équivalents. La règle $\neg_{d}$ ne modifie donc pas la sémantique de la formule. Ainsi, si avant l'application de la règle une formule est valide, elle le sera après l'application. \\

\vspace{10px}

\exo{A(c)}\\
\begin{prooftree}
{\bfseries (4)}
            \AxiomC{$\Gamma \vdash \Phi, A(t)$}
            \RightLabel{\scriptsize$\exists_{d}$}
            \UnaryInfC{$\Gamma \vdash \Phi,\exists.x A(x)$}
\end{prooftree}

\begin{prooftree}
{\bfseries (5)}
            \AxiomC{$\Gamma \vdash \Phi, A(x)$}
            \RightLabel{\scriptsize$\forall_{d}$}
            \UnaryInfC{$\Gamma \vdash \Phi, \forall .x A(x)$}
\end{prooftree}

\begin{prooftree}
{\bfseries (6)}
            \AxiomC{$\Gamma, A(t)\vdash \Phi$}
            \RightLabel{\scriptsize$\forall_{g}$}
            \UnaryInfC{$\Gamma, \forall .x A(x)\vdash \Phi$}
\end{prooftree}

\textbf{(4)}\\
$\llbracket \Gamma \vdash \Phi \lor A(t) \rrbracket^{I}_ {\rho}$\\
$=\llbracket \Gamma \Rightarrow \Phi \lor A(t) \rrbracket^{I}_ {\rho}$\\
$=\llbracket \Gamma \rrbracket^{I}_ {\rho} \Rightarrow_{B} \llbracket \Phi \lor A(t) \rrbracket^{I}_ {\rho}$\\
$=\llbracket \Gamma \rrbracket^{I}_ {\rho} \Rightarrow_{B} (\llbracket \Phi \rrbracket^{I}_ {\rho} \lor_{B}  \llbracket A(t) \rrbracket^{I}_ {\rho})$\\
$=\llbracket \Gamma \rrbracket^{I}_ {\rho} \Rightarrow_{B}( \llbracket \Phi \rrbracket^{I}_ {\rho} \lor_{B} I ( A (\llbracket (t) \rrbracket^{I}_ {\rho})))$\\
$= \llbracket \Gamma \rrbracket^{I}_ {\rho} \Rightarrow_{B} (\llbracket\Phi \rrbracket^{I}_ {\rho} \lor_{B} I(A(I(t))))$\\
$= \llbracket \Gamma \rrbracket^{I}_ {\rho} \Rightarrow_{B} (\llbracket\Phi \rrbracket^{I}_ {\rho} \lor_{B} I(A)(t))$\\
\vspace{10px}

$\Gamma \vdash \Phi,\exists x.A(x)$\\
$= \llbracket \Gamma \Rightarrow \Phi \lor \exists x.A(x) \rrbracket^{I}_ {\rho}$\\
$= \llbracket \Gamma \rrbracket^{I}_ {\rho} \Rightarrow \llbracket \Phi \lor \exists x.A(x) \rrbracket^{I}_ {\rho}$\\
$= \llbracket \Gamma \rrbracket^{I}_ {\rho} \Rightarrow_{B} (\llbracket\Phi \rrbracket^{I}_ {\rho} \lor_{B} \llbracket \exists x.A(x) \rrbracket^{I}_ {\rho})$\\
$= \llbracket \Gamma \rrbracket^{I}_ {\rho} \Rightarrow_{B} (\llbracket\Phi \rrbracket^{I}_ {\rho} \lor_{B} \bigcup_{t \in D} \llbracket A(x) \rrbracket^{I}_ {\rho[t/x]})$\\
$= \llbracket \Gamma \rrbracket^{I}_ {\rho} \Rightarrow_{B} (\llbracket\Phi \rrbracket^{I}_ {\rho} \lor_{B} \bigcup_{t \in D} I(A)\llbracket x \rrbracket^{I}_ {\rho[t/x]})$\\
$= \llbracket \Gamma \rrbracket^{I}_ {\rho} \Rightarrow_{B} (\llbracket\Phi \rrbracket^{I}_ {\rho} \lor_{B} \bigcup_{t \in D} I(A) {\rho[t/x]}(x))$\\
$= \llbracket \Gamma \rrbracket^{I}_ {\rho} \Rightarrow_{B} (\llbracket\Phi \rrbracket^{I}_ {\rho} \lor_{B} \bigcup_{t \in D} I(A)(t))$\\
$= \llbracket \Gamma \rrbracket^{I}_ {\rho} \Rightarrow_{B} (\llbracket\Phi \rrbracket^{I}_ {\rho} \lor_{B} I(A)(t))$\\

\vspace{10px}

On voit donc que les deux séquents du \textbf{(4)} sont sémantiquement équivalents. La règle $\exists_{d}$ ne modifie donc pas la sémantique de la formule. Ainsi, si avant l'application de la règle une formule est valide, elle le sera après l'application. \\

\vspace{10px}

\textbf{(5)}\\
$\llbracket \Gamma \vdash \Phi, A(x) \rrbracket^{I}_ {\rho}$\\
$= \llbracket \Gamma \Rightarrow \Phi \lor A(x) \rrbracket^{I}_ {\rho}$\\
$= \llbracket \Gamma \rrbracket^{I}_ {\rho} \Rightarrow_{B} \llbracket\Phi \lor A(x) \rrbracket^{I}_ {\rho}$\\
$= \llbracket \Gamma \rrbracket^{I}_ {\rho} \Rightarrow_{B} (\llbracket\Phi \lor A(x) \rrbracket^{I}_ {\rho})$\\
$= \llbracket \Gamma \rrbracket^{I}_ {\rho} \Rightarrow_{B} (\llbracket\Phi \rrbracket^{I}_ {\rho} \lor_{B} \llbracket A(x) \rrbracket^{I}_ {\rho})$\\
$= \llbracket \Gamma \rrbracket^{I}_ {\rho} \Rightarrow_{B} (\llbracket\Phi \rrbracket^{I}_ {\rho} \lor_{B} I(A(\llbracket x \rrbracket^{I}_ {\rho})))$\\
$= \llbracket \Gamma \rrbracket^{I}_ {\rho} \Rightarrow_{B} (\llbracket\Phi \rrbracket^{I}_ {\rho} \lor_{B} I(A(I(x))))$\\
$= \llbracket \Gamma \rrbracket^{I}_ {\rho} \Rightarrow_{B} (\llbracket\Phi \rrbracket^{I}_ {\rho} \lor_{B} I(A)(x))$\\
\vspace{10px}

$\llbracket \Gamma \vdash \Phi,\forall x.A(x) \rrbracket^{I}_ {\rho}$\\
$= \llbracket \Gamma \Rightarrow \Phi \lor \forall.x A(x) \rrbracket^{I}_ {\rho}$\\
$= \llbracket \Gamma \rrbracket^{I}_ {\rho} \Rightarrow \llbracket \Phi \lor \forall x.A(x) \rrbracket^{I}_ {\rho}$\\
$= \llbracket \Gamma \rrbracket^{I}_ {\rho} \Rightarrow_{B} (\llbracket\Phi \rrbracket^{I}_ {\rho} \lor_{B} \llbracket \forall x.A(x) \rrbracket^{I}_ {\rho})$\\
$= \llbracket \Gamma \rrbracket^{I}_ {\rho} \Rightarrow_{B} (\llbracket\Phi \rrbracket^{I}_ {\rho} \lor_{B} \bigcap_{x \in D}\llbracket A(x) \rrbracket^{I}_ {\rho[x/x]})$\\
$= \llbracket \Gamma \rrbracket^{I}_ {\rho} \Rightarrow_{B} (\llbracket\Phi \rrbracket^{I}_ {\rho} \lor_{B} \bigcap_{x \in D} I(A) \llbracket x \rrbracket^{I}_ {\rho[x/x]})$\\
$= \llbracket \Gamma \rrbracket^{I}_ {\rho} \Rightarrow_{B} (\llbracket\Phi \rrbracket^{I}_ {\rho} \lor_{B} \bigcap_{x \in D} I(A)\rho[x/x](x))$\\
$= \llbracket \Gamma \rrbracket^{I}_ {\rho} \Rightarrow_{B} (\llbracket\Phi \rrbracket^{I}_ {\rho} \lor_{B} \bigcap_{x \in D} I(A)(x))$\\
$= \llbracket \Gamma \rrbracket^{I}_ {\rho} \Rightarrow_{B} (\llbracket\Phi \rrbracket^{I}_ {\rho} \lor_{B} I(A)(x))$\\

\vspace{10px}

On voit donc que les deux séquents du \textbf{(5)} sont sémantiquement équivalents. La règle $\forall_{d}$ ne modifie donc pas la sémantique de la formule. Ainsi, si avant l'application de la règle une formule est valide, elle le sera après l'application. \\

\vspace{10px}

\textbf{(6)}\\
$\llbracket \Gamma, A(t) \vdash \Phi \rrbracket^{I}_{\rho}$\\
$=\llbracket \Gamma \land A(t) \Rightarrow \Phi \rrbracket^{I}_{\rho}$\\
$=\llbracket \Gamma  \land A(t)\rrbracket^{I}_{\rho} \Rightarrow_{B} \llbracket \Phi \rrbracket^{I}_{\rho}$\\
$=(\llbracket \Gamma \rrbracket^{I}_{\rho} \land_{B} \llbracket A(t)\rrbracket^{I}_{\rho}) \Rightarrow_{B} \llbracket \Phi \rrbracket^{I}_{\rho}$\\
$=(\llbracket \Gamma \rrbracket^{I}_{\rho} \land_{B}  I (A) \llbracket x \rrbracket^{I}_{\rho[t/x]}) \Rightarrow_{B} \llbracket \Phi \rrbracket^{I}_{\rho}$\\
$=(\llbracket \Gamma \rrbracket^{I}_{\rho} \land_{B} I(A)\rho[t/x](x)) \Rightarrow_{B} \llbracket \Phi \rrbracket^{I}_{\rho}$\\
$=(\llbracket \Gamma \rrbracket^{I}_{\rho} \land_{B} I(A)(x)) \Rightarrow_{B} \llbracket \Phi \rrbracket^{I}_{\rho}$\\

\vspace{10px}




$\llbracket \Gamma, \forall x.A(x) \vdash \Phi \rrbracket^{I}_{\rho}$\\
$=\llbracket \Gamma \land \forall x.A(x) \Rightarrow \Phi \rrbracket^{I}_{\rho}$\\
$=\llbracket \Gamma \land \forall x.A(x)\rrbracket^{I}_{\rho} \Rightarrow_{B}\llbracket \Phi \rrbracket^{I}_{\rho}$\\
$=(\llbracket \Gamma \rrbracket^{I}_{\rho}\land_{B} \llbracket\forall x.A(x)\rrbracket^{I}_{\rho}) \Rightarrow_{B}\llbracket \Phi \rrbracket^{I}_{\rho}$\\
$=(\llbracket \Gamma \rrbracket^{I}_{\rho}\land_{B}  \bigcap_{t \in D} \llbracket A(x)\rrbracket^{I}_{\rho[t/x]}) \Rightarrow_{B}\llbracket \Phi \rrbracket^{I}_{\rho}$\\
$=(\llbracket \Gamma \rrbracket^{I}_{\rho}\land_{B}  \bigcap_{t \in D} I (A) \llbracket x \rrbracket^{I}_{\rho[t/x]}) \Rightarrow_{B}\llbracket \Phi \rrbracket^{I}_{\rho}$\\
$=(\llbracket \Gamma \rrbracket^{I}_{\rho}\land_{B}  \bigcap_{t \in D} I(A)\rho[t/x](x)) \Rightarrow_{B}\llbracket \Phi \rrbracket^{I}_{\rho}$\\
$=(\llbracket \Gamma \rrbracket^{I}_{\rho}\land_{B}   I(A)(x)) \Rightarrow_{B}\llbracket \Phi \rrbracket^{I}_{\rho}$\\
\vspace{10px}

On voit donc que les deux séquents du \textbf{(6)} sont sémantiquement équivalents. La règle $\forall_{g}$ ne modifie donc pas la sémantique de la formule. Ainsi, si avant l'application de la règle une formule est valide, elle le sera après l'application. \\

\vspace{10px}

\exo{B(a)}\\
\begin{prooftree}
{\bfseries (7)}
            \AxiomC{$\Gamma, A \vdash \Phi, B$}
            \RightLabel{\scriptsize$\Rightarrow_{d}$}
            \UnaryInfC{$\Gamma \vdash \Phi, A \Rightarrow B$}
\end{prooftree}


\begin{prooftree}
{\bfseries (8)}
            \AxiomC{$\Gamma \vdash \Phi, A$}
            \RightLabel{\scriptsize$\neg_{g}$}
            \UnaryInfC{$\Gamma, \neg A \vdash \Phi$}
\end{prooftree}

\begin{prooftree}
{\bfseries (9)}
            \AxiomC{$\Gamma \vdash \Phi, B$}
            \AxiomC{$\Gamma \vdash \Phi, A$}
            \RightLabel{\scriptsize$\land_{d}$}
            \BinaryInfC{$\Gamma \vdash \Phi, A \land B$}
\end{prooftree}

\textbf{(7)}\\
$\llbracket \Gamma, A \vdash \Phi, B \rrbracket^{I}_ {\rho}$\\
$= \llbracket \Gamma \land A \Rightarrow \Phi \lor B \rrbracket^{I}_ {\rho}$\\
$= \llbracket \Gamma \land A \rrbracket^{I}_ {\rho} \Rightarrow_{B} \llbracket\Phi \lor B \rrbracket^{I}_ {\rho}$\\
$= \llbracket \Gamma \land A \rrbracket^{I}_ {\rho} \Rightarrow_{B} \llbracket\Phi \lor B \rrbracket^{I}_ {\rho}$\\
$= \llbracket \Gamma \rrbracket^{I}_ {\rho} \land_{B} \llbracket A \rrbracket^{I}_ {\rho} \Rightarrow_{B} \llbracket\Phi \rrbracket^{I}_ {\rho} \lor_{B} \llbracket B \rrbracket^{I}_ {\rho}$\\
$= \neg_{B}(\llbracket \Gamma \rrbracket^{I}_ {\rho} \land_{B} \llbracket A \rrbracket^{I}_ {\rho}) \lor_{B} \llbracket\Phi \rrbracket^{I}_ {\rho} \lor_{B} \llbracket B \rrbracket^{I}_ {\rho}$\\
$= \neg_{B}\llbracket \Gamma \rrbracket^{I}_ {\rho} \lor_{B} \neg_{B}\llbracket A \rrbracket^{I}_ {\rho} \lor_{B} \llbracket\Phi \rrbracket^{I}_ {\rho} \lor_{B} \llbracket B \rrbracket^{I}_ {\rho}$\\
\vspace{10px}

$\llbracket \Gamma \vdash \Phi, A \Rightarrow B \rrbracket^{I}_ {\rho}$\\
$= \llbracket \Gamma \Rightarrow \Phi \lor (A \Rightarrow B) \rrbracket^{I}_ {\rho}$\\
$= \llbracket \Gamma \rrbracket^{I}_ {\rho} \Rightarrow \llbracket \Phi \lor (A \Rightarrow B) \rrbracket^{I}_ {\rho}$\\
$= \llbracket \Gamma \rrbracket^{I}_ {\rho} \Rightarrow_{B} \llbracket \Phi \lor (A \Rightarrow B) \rrbracket^{I}_ {\rho}$\\
$= \neg_{B}\llbracket \Gamma \rrbracket^{I}_ {\rho} \lor_{B} \llbracket \Phi \lor (A \Rightarrow B) \rrbracket^{I}_ {\rho}$\\
$= \neg_{B}\llbracket \Gamma \rrbracket^{I}_ {\rho} \lor_{B} \llbracket \Phi \rrbracket^{I}_ {\rho} \lor_{B} (\llbracket A \Rightarrow B \rrbracket^{I}_ {\rho})$\\
$= \neg_{B}\llbracket \Gamma \rrbracket^{I}_ {\rho} \lor_{B} \llbracket \Phi \rrbracket^{I}_ {\rho} \lor_{B} (\llbracket A \rrbracket^{I}_ {\rho} \Rightarrow_{B} \llbracket B \rrbracket^{I}_ {\rho})$\\
$= \neg_{B}\llbracket \Gamma \rrbracket^{I}_ {\rho} \lor_{B} \llbracket \Phi \rrbracket^{I}_ {\rho} \lor_{B} \neg_{B}\llbracket A \rrbracket^{I}_{\rho} \lor_{B} \llbracket B \rrbracket^{I}_ {\rho}$\\
$= \neg_{B}\llbracket \Gamma \rrbracket^{I}_ {\rho} \lor_{B} \neg_{B}\llbracket A \rrbracket^{I}_{\rho} \lor_{B} \llbracket \Phi \rrbracket^{I}_ {\rho} \lor_{B} \llbracket B \rrbracket^{I}_ {\rho}$\\

On voit donc que les deux séquents du \textbf{(7)} sont sémantiquement équivalents. La règle $\Rightarrow_{d}$ ne modifie donc pas la sémantique de la formule. La règle est donc réversible, son application ne modifie pas la validité de la formule. \\


\textbf{(8)}\\
$\llbracket \Gamma \vdash \Phi, A \rrbracket^{I}_ {\rho}$\\
$= \llbracket \Gamma \Rightarrow \Phi \lor A\rrbracket^{I}_ {\rho}$\\
$= \llbracket \Gamma \rrbracket^{I}_ {\rho} \Rightarrow_{B} \llbracket \Phi \lor A\rrbracket^{I}_ {\rho}$\\
$= \llbracket \Gamma \rrbracket^{I}_ {\rho}\Rightarrow_{B} \llbracket \Phi \rrbracket^{I}_ {\rho} \lor_{B} \llbracket A \rrbracket^{I}_ {\rho}  $\\
$= \neg_{B} \llbracket \Gamma \rrbracket^{I}_ {\rho} \lor_{B} \llbracket\Phi \rrbracket^{I}_ {\rho} \lor_{B} \llbracket A \rrbracket^{I}_ {\rho}$\\
\vspace{10px}

$\llbracket \Gamma \land \neg A \vdash \Phi \rrbracket^{I}_ {\rho}$\\
$= \llbracket \Gamma \land \neg  A \Rightarrow  \Phi \rrbracket^{I}_ {\rho}$\\
$= \llbracket \Gamma \land \neg  A\rrbracket^{I}_ {\rho} \Rightarrow_{B} \llbracket \Phi \rrbracket^{I}_ {\rho}$\\
$= (\llbracket \Gamma \rrbracket^{I}_ {\rho}\land_{B} \llbracket \neg A \rrbracket^{I}_ {\rho}) \Rightarrow_{B}  \llbracket\Phi \rrbracket^{I}_ {\rho}$\\
$=  \neg_{B}(\llbracket \Gamma \rrbracket^{I}_ {\rho} \land_{B} \neg_{B} \llbracket A \rrbracket^{I}_ {\rho}) \lor_{B} \llbracket\Phi \rrbracket^{I}_ {\rho}$\\
$= \neg_{B} \llbracket \Gamma \rrbracket^{I}_ {\rho} \lor_{B} \llbracket A \rrbracket^{I}_ {\rho} \lor_{B} \llbracket\Phi \rrbracket^{I}_ {\rho}$\\
$= \neg_{B} \llbracket \Gamma \rrbracket^{I}_ {\rho} \lor_{B} \llbracket\Phi \rrbracket^{I}_ {\rho} \lor_{B} \llbracket A \rrbracket^{I}_ {\rho}$\\
\vspace{10px}

On voit donc que les deux séquents du \textbf{(8)} sont sémantiquement équivalents. La règle $\neg_{g}$ ne modifie donc pas la sémantique de la formule. La règle est donc réversible, son application ne modifie pas la validité de la formule. \\

\vspace{10px}
\textbf{(9)}\\
 $\llbracket \Gamma \vdash \Phi, B  \rrbracket^{I}_{\rho} \land_{B} \llbracket\Gamma \vdash \Phi, A \rrbracket^{I}_{\rho}$\\
 $=\llbracket \Gamma \Rightarrow (\Phi \lor B ) \rrbracket^{I}_{\rho} \land_{B} \llbracket\Gamma \Rightarrow( \Phi\lor A) \rrbracket^{I}_{\rho}$\\
 $=(\llbracket \Gamma \rrbracket^{I}_{\rho} \Rightarrow_{B} \llbracket \Phi \lor B \rrbracket^{I}_{\rho}) \land_{B} \llbracket\Gamma \Rightarrow( \Phi\lor A) \rrbracket^{I}_{\rho}$\\
 $= (\llbracket \Gamma \rrbracket^{I}_{\rho} \Rightarrow_{B} (\llbracket \Phi \rrbracket^{I}_{\rho} \lor_{B} \llbracket B \rrbracket^{I}_{\rho})) \land_{B} \llbracket\Gamma \Rightarrow( \Phi\lor A) \rrbracket^{I}_{\rho}$\\
$= (\llbracket \Gamma \rrbracket^{I}_{\rho} \Rightarrow_{B} (\llbracket \Phi \rrbracket^{I}_{\rho} \lor_{B} \llbracket B \rrbracket^{I}_{\rho})) \land_{B} (\llbracket \Gamma \rrbracket^{I}_{\rho} \Rightarrow_{B} \llbracket \Phi\lor A \rrbracket^{I}_{\rho})$\\
$=(\llbracket \Gamma \rrbracket^{I}_{\rho} \Rightarrow_{B} (\llbracket \Phi \rrbracket^{I}_{\rho} \lor_{B} \llbracket B \rrbracket^{I}_{\rho})) \land_{B} (\llbracket \Gamma \rrbracket^{I}_{\rho} \Rightarrow_{B} (\llbracket \Phi\rrbracket^{I}_{\rho} \lor_{B} \llbracket A \rrbracket^{I}_{\rho}))$\\
$= (\neg_{B} \llbracket \Gamma \rrbracket^{I}_{\rho} \lor_{B} \llbracket \Phi \rrbracket^{I}_{\rho} \lor_{B} \llbracket B \rrbracket^{I}_{\rho}) \land_{B} (\neg_{B}\llbracket \Gamma \rrbracket^{I}_{\rho} \lor_{B} \llbracket \Phi \rrbracket^{I}_{\rho} \lor_{B} \llbracket A \rrbracket^{I}_{\rho})$\\
$= \neg_{B} \llbracket \Gamma \rrbracket^{I}_{\rho} \lor_{B} \llbracket \Phi \rrbracket^{I}_{\rho} \lor_{B} (\llbracket A \rrbracket^{I}_{\rho} \land_{B} B \rrbracket^{I}_{\rho})$\\
\vspace{10px}

$\llbracket \Gamma \vdash \Phi, A \land B \rrbracket^{I}_{\rho}$\\
$= \llbracket \Gamma \Rightarrow \Phi, A \land B \rrbracket^{I}_{\rho}$\\
$= \llbracket \Gamma \Rightarrow \Phi \lor (A \land B) \rrbracket^{I}_{\rho}$\\
$= \llbracket \Gamma \rrbracket^{I}_{\rho} \Rightarrow_{B} (\llbracket \Phi \lor (A \land B) \rrbracket^{I}_{\rho})$\\
$= \neg_{B} \llbracket \Gamma \rrbracket^{I}_{\rho} \lor_{B} \llbracket \Phi \lor (A \land B) \rrbracket^{I}_{\rho})$\\
$= \neg_{B} \llbracket \Gamma \rrbracket^{I}_{\rho} \lor_{B} (\llbracket \Phi \rrbracket^{I}_{\rho} \lor_{B} (\llbracket A \land B \rrbracket^{I}_{\rho}))$\\
$= \neg_{B} \llbracket \Gamma \rrbracket^{I}_{\rho} \lor_{B} \llbracket \Phi \rrbracket^{I}_{\rho} \lor_{B} (\llbracket A \rrbracket^{I}_{\rho} \land_{B} \llbracket B \rrbracket^{I}_{\rho})$\\

\vspace{10px}

On voit donc que les deux séquents du \textbf{(9)} sont sémantiquement équivalents. La règle $\land_{d}$ ne modifie donc pas la sémantique de la formule. La règle est donc réversible, son application ne modifie pas la validité de la formule. \\


\end{document}