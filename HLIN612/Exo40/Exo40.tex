\documentclass[11pt,letterpaper]{article}
\usepackage{fullpage}
\usepackage[top=1.5cm, bottom=3.5cm, left=1.5cm, right=1.5cm]{geometry}
\usepackage{amsmath,amsthm,amsfonts,amssymb,amscd}
\usepackage{french}
\usepackage[utf8]{inputenc}
\usepackage[T1]{fontenc}
\usepackage{lastpage}
\usepackage{enumerate}
\usepackage{fancyhdr}
\usepackage{mathrsfs}
\usepackage{xcolor}

\usepackage{graphics} %inclusion de figures
\usepackage{graphicx} %inclusion de figures

\usepackage{sidecap}
\sidecaptionvpos{figure}{c}
\usepackage{listings}
\usepackage{hyperref}
\usepackage{numprint}
\usepackage[bottom]{footmisc}
\usepackage[justification=centering]{caption}
\usepackage[french,frenchkw,boxed,ruled,lined]{algorithm2e}
\SetKw{KwDe}{de}
\SetKw{KwOu}{ou}
\SetKw{KwEt}{et}
\SetKw{Kwrenv}{renvoyer}
\SetKwInput{Variables}{Variables}
\SetKwInput{Variable}{Variable}
\usepackage{xifthen}
\usepackage{colortbl}
\definecolor{green}{rgb}{0.0, 0.5, 0.0}
\usepackage{array,multirow}
\usepackage{wrapfig}
\usepackage{bussproofs}
\usepackage{tikz}
\usetikzlibrary{positioning}
\definecolor{processblue}{cmyk}{0.96,0,0,0}

\newcommand{\exo}[1]{\Large \textbf{Exercice \numprint{#1}} \vspace{10px} \normalsize}
\newcommand\tab[1][12pt]{\hspace*{#1}}

\newcommand{\titlebox}[2]{%
\tikzstyle{titlebox}=[rectangle,inner sep=10pt,inner ysep=10pt,draw]%
\tikzstyle{title}=[fill=white]%
%
\bigskip\noindent\begin{tikzpicture}
\node[titlebox] (box){%
    \begin{minipage}{0.94\textwidth}
#2
    \end{minipage}
};
%\draw (box.north west)--(box.north east);
\node[title] at (box.north) {#1};
\end{tikzpicture}\bigskip%
}

% INFOS
\author{Antoine AFFLATET et Jérémie ROUX (L3 Groupe C)}
\title{HLIN612 - Calculabilité/Complexité \\Exercice 40 - Programmation dynamique : algorithme pseudo-polynomial}
\date{2019 - 2020}

\setlength{\parindent}{0cm}

\begin{document}

\maketitle

\exo{1}\\

\vspace{-20px}

\titlebox{Problème de \textsc{2-Partition}}{
\vspace{3px}
\textbf{Entrée :} Étant donnés $n$ objets $a_i$ ($1 \leq i \leq n$) de poids entiers $p(a_1), p(a_2), ..., p(a_n)$ de somme $2P$.\\
\textbf{Question :} Est-il possible de les partager en 2 sous-ensembles de même poids total $P$ ?
}

On considère $i$ et $j$, tels que $1 \leq i \leq n$ et $0 \leq j \leq P$ et $T(i,j)$ : ``Étant donnés les $i$ premiers éléments de la famille, il existe un sous-ensemble de ces $i$ éléments de poids $j$''.\\

$n=6$\\
$2P = \sum\limits_{i=1}^n p(a_i) = \sum\limits_{i=1}^6 p(a_i) = p(a_1) + p(a_2) + p(a_3) + p(a_4) + p(a_5) + p(a_6) = 5+9+3+8+2+5=32$\\
donc $P = \frac{32}{2} = 16$\\

 T[i,j] est vrai si et seulement si ($p(a_i) == j)$ || ($T[i,j] == T[i-1,j]$) || ($T[i-1,j-p(a_i)]==1$)) l'est aussi.\\

\begin{tabular}{|c|c|c|c|c|c|c|c|c|c|c|c|c|c|c|c|c|c|}
    \hline
    $i/j$ & $0$ & $1$ & $2$ & $3$ & $4$ & $5$ & $6$ & $7$ & $8$ & $9$ & $10$ & $11$ & $12$ & $13$ & $14$ & $15$ & $16$\\
    \hline
    
    1 & \cellcolor{green!50} & & & & & \cellcolor{green!50}& & & & & & & & & & &\\
    \hline
    2 & \cellcolor{green!50} & & & & & \cellcolor{green!50}& & & & \cellcolor{green!50}& & & & &\cellcolor{green!50} & &\\
    \hline
    3 & \cellcolor{green!50} & & &\cellcolor{green!50} & & \cellcolor{green!50}& & &\cellcolor{green!50} & \cellcolor{green!50}& & &\cellcolor{green!50} & & \cellcolor{green!50}& &\\
    \hline
    4 & \cellcolor{green!50} & & &\cellcolor{green!50} & & \cellcolor{green!50}& & &\cellcolor{green!50} & \cellcolor{green!50}& &  \cellcolor{green!50}&\cellcolor{green!50} &  \cellcolor{green!50} & \cellcolor{green!50}& & \cellcolor{green!50}\\
    \hline
    5 & \cellcolor{green!50} & &\cellcolor{green!50} &\cellcolor{green!50} & & \cellcolor{green!50}& &\cellcolor{green!50} &\cellcolor{green!50} & \cellcolor{green!50}& \cellcolor{green!50} &  \cellcolor{green!50}&\cellcolor{green!50} &  \cellcolor{green!50} & \cellcolor{green!50}& \cellcolor{green!50} & \cellcolor{green!50}\\
    \hline
    6 & \cellcolor{green!50} & &\cellcolor{green!50} &\cellcolor{green!50} & & \cellcolor{green!50}& &\cellcolor{green!50} &\cellcolor{green!50} & \cellcolor{green!50}& \cellcolor{green!50} &  \cellcolor{green!50}&\cellcolor{green!50} &  \cellcolor{green!50} & \cellcolor{green!50}& \cellcolor{green!50} & \cellcolor{green!50}\\
    \hline
\end{tabular}

\vspace{20px}

\begin{tabular}{|c|c|c|c|c|c|c|c|c|c|c|c|c|c|c|c|c|c|}
    \hline
    $i/j$ & $0$ & $1$ & $2$ & $3$ & $4$ & $5$ & $6$ & $7$ & $8$ & $9$ & $10$ & $11$ & $12$ & $13$ & $14$ & $15$ & $16$\\
    \hline
    1 & $\{\{\}\}$ & $\emptyset$ & $\emptyset$ & $\emptyset$ & $\emptyset$ & $\{\{a_1\}\}$ & $\emptyset$ & $\emptyset$ & $\emptyset$ & $\emptyset$ & $\emptyset$ & $\emptyset$ & $\emptyset$ & $\emptyset$ & $\emptyset$ & $\emptyset$ & $\emptyset$ \\
    \hline
    2 & $\{\{\}\}$ & $\emptyset$ & $\emptyset$ & $\emptyset$ & $\emptyset$ & $\{\{a_1\}\}$ & $\emptyset$ & $\emptyset$ & $\emptyset$ & $\{a_2\}$ & $\emptyset$ & $\emptyset$ & $\emptyset$ & $\emptyset$ & $\emptyset$ & $\emptyset$ & $\emptyset$ \\
    \hline
    3 & $\{\{\}\}$ & $\emptyset$ & $\emptyset$ & $\{\{a_3\}\}$ & $\emptyset$ & $\{\{a_1\}\}$ & $\emptyset$ & $\emptyset$ & $\{\{a_1,a_3\}\}$ & $\{a_2\}$ & $\emptyset$ & $\emptyset$ & $\emptyset$ & $\emptyset$ & $\emptyset$ & $\emptyset$ & $\emptyset$ \\
    \hline
    4 & $\{\{\}\}$ & $\emptyset$ & $\emptyset$ & $\{\{a_3\}\}$ & $\emptyset$ & $\{\{a_1\}\}$ & $\emptyset$ & $\emptyset$ & $\{\{a_1,a_3\},\{a_4\}\}$ & $\{a_2\}$ & $\emptyset$ & $\emptyset$ & $\emptyset$ & $\emptyset$ & $\emptyset$ & $\emptyset$ & $\emptyset$ \\
    \hline
    5 & $\{\{\}\}$ & $\emptyset$ & $\{\{a_5\}\}$ & $\{\{a_3\}\}$ & $\emptyset$ & $\{\{a_1\}\},\{a_3,a_5\}\}$ & $\emptyset$ & $\{\{a_1,a_3\}\}$ & $\{\{a_1,a_3\},\{a_4\}\}$ & $\{a_2\}$ & $\{\{a_1,a_3,a_5\}\}$ & $\emptyset$ & $\emptyset$ & $\emptyset$ & $\emptyset$ & $\emptyset$ & $\emptyset$ \\
    \hline
    6 & $\{\{\}\}$ & $\emptyset$ & $\{\{a_5\}\}$ & $\{\{a_3\}\}$ & $\emptyset$ & $\{\{a_1\}\},\{a_3,a_5\},\{a_6\}\}$ & $\emptyset$ & $\{\{a_1,a_5\},\{a_6,a_5\}\}$ & $\{\{a_1,a_3\},\{a_4\},\{a_6,a_3\}\}$ & $\{a_2\}$ & $\{\{a_1,a_3,a_5\},\{a_1,a_6\},\{a_6,a_3,a_5\}\}$ & $\emptyset$ & $\emptyset$ & $\emptyset$ & $\emptyset$ & $\emptyset$ & $\emptyset$ \\
    \hline
\end{tabular}

Depuis ce tableau (tableau 1) on prend l'élément de la première ligne vérifiant la colonne 16, puis celui de la première ligne vérifiant $16-p(a_i)$ avec $a_i$ étant l'élément de la première ligne vérifiant la colonne 16. On reproduit jusqu'à avoir tous les éléments.\\
  
\begin{algorithm}[H]
    $i \gets n$\;
    $j \gets P$\;
    $S \gets \{\}$\;
    \Tq{$i>0$ et $T[i][j] \neq 0$}{
        $i--$\; 
        $S \gets S \cup \{a_{i+1}\}$\;
        $j \gets j - p(a_{i+1})$\;
        $i \gets n$\;
    }
    \textbf{renvoyer} $S$\;
    
    \caption{Reconstruction de la solution($T$ : tableau à 2 dimensions, $P$ : poids total, $n$ : entier) : ensemble}
\end{algorithm}

Complexité de l'algorithme = $O(n*P)$ où n est le nombre d'élément dans l'ensemble et $P= 1/2*\sum\limits_{i=1}^n p(a_i)$


\newpage
\exo{2}\\

\vspace{-20px}

\titlebox{Problème du sac à dos sans répétition}{
\vspace{3px}
Les objets seront pris au plus 1 fois. Pour cela considérons, un tableau $K$ à deux dimensions tel que $K[j,w]$ représente la valeur maximale que l'on peut stocker dans un sac de capacité $w$, avec des objets $1, .., j$.
}

$K[0,w] = 0$ et $K[j,0] = 0$\\
$K[j,w] =$ max\\

$K[j,w] =$ max($K[j - 1,w],K[j - 1,w - w_j] + v_j$)\\

\begin{tabular}{|c|ccccccccccccc|}
\hline
$j\backslash w$&$0$&$1$&$2$&$3$&$4$&$5$&$6$&$7$&$8$&$9$&$10$&$11$&$12$\\
\hline
$0$&$0$&$0$&$0$&$0$&$0$&$0$&$0$&$0$&$0$&$0$&$0$&$0$&$0$\\
\hline
$1$&$0$&$1$&$1$&$1$&$1$&$1$&$1$&$1$&$1$&$1$&$1$&$1$&$1$\\
\hline
$2$&$\mathbf{0}$&$1$&$6$&$7$&$7$&$7$&$7$&$7$&$7$&$7$&$7$&$7$&$7$\\
\hline
$3$&$0$&$1$&$6$&$7$&$7$&$\mathbf{18}$&$19$&$24$&$25$&$25$&$25$&$25$&$25$\\
\hline
$4$&$0$&$1$&$6$&$7$&$7$&$\mathbf{18}$&$22$&$24$&$28$&$29$&$29$&$40$&$41$\\
\hline
$5$&$0$&$1$&$6$&$7$&$7$&$18$&$22$&$24$&$28$&$30$&$31$&$40$&$\mathbf{42}$\\
\hline
\end{tabular}

\end{document}